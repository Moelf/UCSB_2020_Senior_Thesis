%%
%% This is file `sampleEqPg.tex',
%% generated with the docstrip utility.
%%
%% The original source files were:
%%
%% glossary.dtx  (with options: `sampleEqPg.tex,package')
%% Copyright (C) 2006 Nicola Talbot, all rights reserved.
%% If you modify this file, you must change its name first.
%% You are NOT ALLOWED to distribute this file alone. You are NOT
%% ALLOWED to take money for the distribution or use of either this
%% file or a changed version, except for a nominal charge for copying
%% etc.
%% \CharacterTable
%%  {Upper-case    \A\B\C\D\E\F\G\H\I\J\K\L\M\N\O\P\Q\R\S\T\U\V\W\X\Y\Z
%%   Lower-case    \a\b\c\d\e\f\g\h\i\j\k\l\m\n\o\p\q\r\s\t\u\v\w\x\y\z
%%   Digits        \0\1\2\3\4\5\6\7\8\9
%%   Exclamation   \!     Double quote  \"     Hash (number) \#
%%   Dollar        \$     Percent       \%     Ampersand     \&
%%   Acute accent  \'     Left paren    \(     Right paren   \)
%%   Asterisk      \*     Plus          \+     Comma         \,
%%   Minus         \-     Point         \.     Solidus       \/
%%   Colon         \:     Semicolon     \;     Less than     \<
%%   Equals        \=     Greater than  \>     Question mark \?
%%   Commercial at \@     Left bracket  \[     Backslash     \\
%%   Right bracket \]     Circumflex    \^     Underscore    \_
%%   Grave accent  \`     Left brace    \{     Vertical bar  \|
%%   Right brace   \}     Tilde         \~}
\documentclass[a4paper,12pt]{report}

\usepackage{amsmath}
\usepackage[colorlinks]{hyperref}
\usepackage[header,toc,border=none,cols=3,
            number=equation]{glossary}[2006/07/20]

\newcommand{\erf}{\operatorname{erf}}
\newcommand{\erfc}{\operatorname{erfc}}

\renewcommand{\glossaryname}{Index of Special Functions and Notations}

\renewcommand{\glossarypreamble}{Numbers in italic indicate the equation number,
numbers in bold indicate page numbers where the main definition occurs.\par}

\setglossary{glsnumformat=hyperit}

\renewcommand{\entryname}{Notation}

\renewcommand{\descriptionname}{Function Name}

\newcommand{\glossarysubheader}{ & & \\}


\storeglosentry{Gamma}{name=\ensuremath{\Gamma(z)},
description=Gamma function,sort=Gamma}

\storeglosentry{gamma}{name=\ensuremath{\gamma(\alpha,x)},
description=Incomplete gamma function,sort=gamma}

\storeglosentry{iGamma}{name=\ensuremath{\Gamma(\alpha,x)},
description=Incomplete gamma function,sort=Gamma}

\storeglosentry{psi}{name=\ensuremath{\psi(x)},
description=Psi function,sort=psi}

\storeglosentry{erf}{name=\ensuremath{\erf(x)},
description=Error function,sort=erf}

\storeglosentry{erfc}{name=\ensuremath{\erfc(x)},
description=Complementary error function,sort=erfc}

\storeglosentry{beta}{name=\ensuremath{B(x,y)},
description=Beta function,sort=B}

\storeglosentry{Bx}{name=\ensuremath{B_x(p,q)},
description=Incomplete beta function,sort=Bx}

\storeglosentry{Tn}{name=\ensuremath{T_n(x)},
description=Chebyshev's polynomials of the first kind,
sort=Tn}

\storeglosentry{Un}{name=\ensuremath{U_n(x)},
description=Chebyshev's polynomials of the second kind,
sort=Un}

\storeglosentry{Hn}{name=\ensuremath{H_n(x)},
description=Hermite polynomials,sort=Hn}

\storeglosentry{Lna}{name=\ensuremath{L_n^\alpha(x)},
description=Laguerre polynomials,sort=Lna}

\storeglosentry{Znu}{name=\ensuremath{Z_\nu(z)},
description=Bessel functions,sort=Z}

\storeglosentry{Pagz}{name=\ensuremath{\Phi(\alpha,\gamma;z)},
description=confluent hypergeometric function,sort=Pagz}

\storeglosentry{kv}{name=\ensuremath{k_\nu(x)},
description=Bateman's function,sort=kv}

\storeglosentry{Dp}{name=\ensuremath{D_p(z)},
description=Parabolic cylinder functions,sort=Dp}

\storeglosentry{Fpk}{name=\ensuremath{F(\phi,k)},
description=Elliptical integral of the first kind,sort=Fpk}

\storeglosentry{C}{name=\ensuremath{C},
description=Euler's constant,sort=C}

\storeglosentry{G}{name=\ensuremath{G},
description=Catalan's constant,sort=G}

\renewcommand{\shortglossaryname}{Special Functions}

\makeglossary

\pagestyle{headings}

\begin{document}

\title{Sample Document Using Interchangable Numbering}
\author{Nicola Talbot}
\maketitle

\begin{abstract}
This is a sample document illustrating the use of the \textsf{glossary}
package.  The functions here have been taken from ``Tables of
Integrals, Series, and Products'' by I.S.~Gradshteyn and I.M~Ryzhik.

The glossary lists both page numbers and equation numbers.
Since the majority of the entries use the equation number,
\texttt{number=equation} was used as a package option.
The entries that should refer to the page number instead
use the \texttt{number=equation} glossary key.
Note that this example will only work where the
page number and equation number compositor is the same. So
it won't work if, say, the page numbers are of the form
2-4 and the equation numbers are of the form 4.6.
As most of the glossary entries should have an italic
format, it is easiest to set the default format to
italic.

\end{abstract}

\tableofcontents

\printglossary

\chapter{Gamma Functions}

The \useGlosentry[number=page,format=hyperbf]{Gamma}{gamma function} is
defined as
\begin{equation}
\gls{Gamma} = \int_{0}^{\infty}e^{-t}t^{z-1}\,dt
\end{equation}

\begin{equation}
\useGlosentry{Gamma}{\ensuremath{\Gamma(x+1)}} = x\Gamma(x)
\end{equation}

\begin{equation}
\gls{gamma} = \int_0^x e^{-t}t^{\alpha-1}\,dt
\end{equation}

\begin{equation}
\gls{iGamma} = \int_x^\infty e^{-t}t^{\alpha-1}\,dt
\end{equation}

\newpage

\begin{equation}
\useGlosentry{Gamma}{\ensuremath{\Gamma(\alpha)}} = \Gamma(\alpha, x) + \gamma(\alpha, x)
\end{equation}

\begin{equation}
\gls{psi} = \frac{d}{dx}\ln\Gamma(x)
\end{equation}

\chapter{Error Functions}

The \useGlosentry[number=page,format=hyperbf]{erf}{error
function} is defined as:
\begin{equation}
\gls{erf} = \frac{2}{\surd\pi}\int_0^x e^{-t^2}\,dt
\end{equation}

\begin{equation}
\gls{erfc} = 1 - \erf(x)
\end{equation}

\chapter{Beta Function}

\begin{equation}
\gls{beta} = 2\int_0^1 t^{x-1}(1-t^2)^{y-1}\,dt
\end{equation}
Alternatively:
\begin{equation}
\gls{beta} = 2\int_0^{\frac\pi2}\sin^{2x-1}\phi\cos^{2y-1}\phi\,d\phi
\end{equation}

\begin{equation}
\gls{beta} = \frac{\Gamma(x)\Gamma(y)}{\Gamma(x+y)} = B(y,x)
\end{equation}

\begin{equation}
\gls{Bx} = \int_0^x t^{p-1}(1-t)^{q-1}\,dt
\end{equation}

\chapter{Chebyshev's polynomials}

\begin{equation}
\gls{Tn} = \cos(n\arccos x)
\end{equation}

\begin{equation}
\gls{Un} = \frac{\sin[(n+1)\arccos x]}{\sin[\arccos x]}
\end{equation}

\chapter{Hermite polynomials}

\begin{equation}
\gls{Hn} = (-1)^n e^{x^2} \frac{d^n}{dx^n}(e^{-x^2})
\end{equation}

\chapter{Laguerre polynomials}

\begin{equation}
\gls{Lna} = \frac{1}{n!}e^x x^{-\alpha}
\frac{d^n}{dx^n}(e^{-x}x^{n+\alpha})
\end{equation}

\chapter{Bessel Functions}

Bessel functions $Z_\nu(z)$ are solutions of
\begin{equation}
\frac{d^2Z_\nu}{dz^2} + \frac{1}{z}\,\frac{dZ_\nu}{dz} +
\left(
1-\frac{\nu^2}{z^2}Z_\nu = 0
\right)
\end{equation}
\useglosentry{Znu}

\chapter{Confluent hypergeometric function}

\begin{equation}
\gls{Pagz} = 1 + \frac{\alpha}{\gamma}\,\frac{z}{1!}
+ \frac{\alpha(\alpha+1)}{\gamma(\gamma+1)}\,\frac{z^2}{2!}
+\frac{\alpha(\alpha+1)(\alpha+2)}
      {\gamma(\gamma+1)(\gamma+2)}
\,\frac{z^3}{3!}
+ \cdots
\end{equation}

\begin{equation}
\gls{kv} = \frac{2}{\pi}\int_0^{\pi/2}
\cos(x \tan\theta - \nu\theta)\,d\theta
\end{equation}

\chapter{Parabolic cylinder functions}

\begin{equation}
\gls{Dp} = 2^{\frac{p}{2}}e^{-\frac{z^2}{4}}
\left\{
\frac{\surd\pi}{\Gamma\left(\frac{1-p}{2}\right)}
\Phi\left(-\frac{p}{2},\frac{1}{2};\frac{z^2}{2}\right)
-\frac{\sqrt{2\pi}z}{\Gamma\left(-\frac{p}{2}\right)}
\Phi\left(\frac{1-p}{2},\frac{3}{2};\frac{z^2}{2}\right)
\right\}
\end{equation}

\chapter{Elliptical Integral of the First Kind}

\begin{equation}
\gls{Fpk} = \int_0^\phi
\frac{d\alpha}{\sqrt{1-k^2\sin^2\alpha}}
\end{equation}

\chapter{Constants}

\begin{equation}
\gls{C} = 0.577\,215\,664\,901\ldots
\end{equation}

\begin{equation}
\gls{G} = 0.915\,965\,594\ldots
\end{equation}

\end{document}
\endinput
%%
%% End of file `sampleEqPg.tex'.
